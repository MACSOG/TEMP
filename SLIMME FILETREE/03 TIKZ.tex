\documentclass{article}
\usepackage[dutch]{babel}
\usepackage{tikz}
\usetikzlibrary{chains, shapes.misc, positioning, arrows.meta}

\begin{document}

\section*{Dossier: De Eindeloze Pingpong}

\begin{center}
\begin{tikzpicture}[
    % Instellingen voor de hele ketting
    start chain=going below,    % Alles wat je toevoegt komt ONDER de vorige
    node distance=8mm,          % Afstand tussen blokken
    every join/.style={->, thick, >=latex}, % Pijl stijl tussen blokken
    
    % Onze stijlen (dezelfde als eerst)
    base/.style={
        on chain,               % Zorgt dat het in de ketting komt
        join,                   % Verbindt automatisch met de vorige!
        draw, 
        rounded corners, 
        align=center, 
        text width=8cm,         % Vaste breedte zodat het netjes lijnt
        inner sep=6pt
    },
    tegenpartij/.style={base, fill=red!10, draw=red},
    klaar/.style={base, fill=green!10, draw=green!50!black, dashed},
    actie/.style={base, fill=yellow!20, draw=orange, double, font=\bfseries}
]

    % --- HIER BEGINT JE LIJST (GEEN NESTING MEER!) ---
    
    % Stap 1
    \node [base, fill=blue!10] {
        \textbf{Start Dossier}\\
        Cliënt meldt conflict
    };

    % Stap 2
    \node [tegenpartij] {
        \textbf{2023-01-10}\\
        Brief Orde van Advocaten (Klacht)
    };

    % Stap 3
    \node [klaar] {
        \textbf{2023-01-12}\\
        Ontvangstbevestiging gestuurd
    };

    % Stap 4 - Je kunt gewoon door blijven typen onder elkaar
    \node [tegenpartij] {
        \textbf{2023-02-01}\\
        Wederpartij stuurt 400 pagina's onzin
    };

    % Stap 5
    \node [actie] {
        \textbf{DEADLINE 2023-02-14}\\
        Verweerschrift indienen!
    };
    
    % Stap 6 - Enzovoort... (Geen tabs nodig!)
    \node [klaar] {
        \textbf{2023-02-13}\\
        Concept gemaakt
    };

\end{tikzpicture}
\end{center}

\end{document}