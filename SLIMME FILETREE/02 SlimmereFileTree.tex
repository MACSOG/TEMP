\documentclass{article}
\usepackage[dutch]{babel}
\usepackage[dvipsnames]{xcolor}
\usepackage{forest}

% Kleurcodes blijven hetzelfde (handig voor je overzicht)
\tikzset{
  tegenpartij/.style={fill=red!10, draw=red, thick},
  actie/.style={fill=yellow!20, draw=orange, double, font=\bfseries},
  klaar/.style={fill=green!10, draw=green!50!black, dashed},
  basis/.style={draw, rounded corners, align=center, inner sep=6pt}
}

\begin{document}

\section*{Juridische Pingpong (Chronologisch)}

\begin{forest}
  for tree={
    grow=south,           % HIER: Alles groeit recht naar beneden
    basis,                % Pas de basis opmaak toe op alles
    l sep=1.5cm,          % Verticale afstand tussen de blokken
    edge={->, >=latex},   % Pijltjes toevoegen (richting)
    where n children=0{   % De laatste stap iets benadrukken? (Optioneel)
      tier=end            % Zorgt dat eindes op gelijke hoogte komen (als je vertakt)
    }{}
  }
  [Start Dossier: Conflict Aannemer, basis
    [2023-01-10: Ingebrekestelling ontvangen, tegenpartij
      [2023-01-12: Gespreksnotitie bellen, klaar
        [2023-01-20: Verweerbrief verstuurd, klaar
          [2023-02-05: Reactie: Ze ontkennen alles, tegenpartij
            % Hier splitst het zich: je hebt twee opties of sporen
            [Optie A: Schikken (Brief sturen), actie]
            [Optie B: Procederen (Advocaat bellen), actie]
          ]
        ]
      ]
    ]
  ]
\end{forest}

\end{document}