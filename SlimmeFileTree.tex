\documentclass{article}
\usepackage[dutch]{babel}
\usepackage[dvipsnames]{xcolor}
\usepackage{forest}

% We definiëren stijlen voor de pingpong
\tikzset{
  % Stijl voor 'Tegenpartij' (Rood/Gevaar)
  tegenpartij/.style={fill=red!10, draw=red, thick},
  % Stijl voor 'Mijn actie' (Geel/Let op!)
  actie/.style={fill=yellow!20, draw=orange, double, font=\bfseries},
  % Stijl voor 'Afgehandeld' (Groen/Rust)
  klaar/.style={fill=green!10, draw=green!50!black, dashed}
}

\begin{document}

\section*{Dossier: De Pingpong}

\begin{forest}
  for tree={
    grow'=0,              % Groeit naar rechts
%    grow'=south,              % Groeit naar rechts
    folder,               % MAAKT HET EEN FILETREE!
    fit=band,             % Zorgt dat tekst past
    s sep=0.2cm,          % Verticale afstand tussen regels
    draw,                 % Teken lijnen om de tekstvakjes
    rounded corners,      % Ronde hoeken zijn vriendelijker
    node options={align=left}, 
  }
  [Dossier X (Start)
    [2023-01-10: Brief Tegenpartij (Eis) , tegenpartij
      [2023-01-14: Mijn verweer (Concept) , klaar]
      [2023-01-15: Mijn verweer (Verzonden) , klaar]
    ]
    [2023-02-01: Reactie Tegenpartij (Nieuw bewijs) , tegenpartij
       [2023-02-02: Email advocaat (Vraag om uitstel) , klaar]
    ]
    [2023-03-10: Dagvaarding ontvangen , tegenpartij
       [NOG TE DOEN: Verweerschrift schrijven , actie]
       [NOG TE DOEN: Bewijsstukken verzamelen , actie]
    ]
  ]
\end{forest}

\end{document}